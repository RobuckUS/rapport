\section{Bilan de la conception}

\todoGeneric{1 à 2 pages de bilan}

\subsection{Comportement du prototype}

Notre robot est muni d’un canon muni d’un moteur DC sans balai servant à propulser une balle de ping-pong.
Un servomoteur fait avancer la crémaillère qui pousse la balle vers la roue pour la tirer en direction des verres.
Le robot peut se déplacer de gauche à droite pour ajuster la direction du tir et la vitesse du moteur peut être augmenté ou diminuer pour ajuster la distance du tir.
Le tout est contrôlé par un téléphone cellulaire à travers un application Android qui communique avec le robot via le protocol Bluetooth.
Un bouton sur l’interface sert aussi à activer les deux moteurs tirer la balle lorsque désiré.
Les verres sont situés plus loin sur une plateforme intelligente qui détecte dans quel verre la balle est entrée.
Pour ce faire, une DEL infrarouge et une photodiode sont situé une en face de l’autre dans chacun des verres et ainsi lorsqu’une balle coupe le faisceau la tension de sortie de la photodiode est plus faible.
Six DELs bleus sont situées sur le dessus de la plateforme pour informer l’utilisateur dans quel verre la balle est rentrée.

\subsection{Différences avec la planification}

La principale différence avec ce que nous avions planifié est qu’il n’y a finalement pas de communication entre la plateforme de verres et le robot.
Ainsi, le robot ne peut pas savoir lorsqu’il a réussi un tir et ne peut donc pas célébrer en dansant comme planifié au début de la session.
De plus, le robot devait être contrôler à l’aide d’une manette que nous devions concevoir, mais finalement il est contrôlé via une application Android.
Nous avons également laissé tomber le mode autonome du robot pour se déplacer et tirer les balles.

\subsection{Efforts demandé}

Malgré le fait que nous n'avons pas inscrit le nombre exacte d'heures passé sur le projet, nous estimons que le projet nous a demandé environ 250 heures réparties en 7 semaines afin de produire le prototype final. Par ailleurs le coût total du projet s'élève à 133~$ ce qui fait un peu moins de 20~$ par personne.

\subsection{Réussites}

Un de nos points forts est que nous avons séparé les tâches dès le début de la session pour faire en sorte que certaines personnes travaillent sur le projet final dès les premières semaines. De cette manière, à la fin des deux défis, nous avions déjà une bonne partie du projet final de terminé ce qui nous a permis de livrer un prototype fini. De plus, nous avons divisé les tâches selon les forces de chacun. Du côté du prototype, un de ses points forts était la facilité d'utilisation. Au cours de plusieurs essaies, nous avons constaté que peu importe la génération de la personne utilisant le robot, celle-ci était en mesure, après quelques minutes, d'utiliser efficacement et simplement l'application conçue. De plus, un des buts du robot était de remplacer un joueur de bière-pong trop affecté par les effets de l'alcool afin de gagner une partie et donc de s'amuser encore plus. De ce côté, nous avons remarquer que notre prototype apportais beaucoup d'amusement aux gens l'utilisant ce qui répond bien au but fixer au début. 

Un des points à améliorer du projet est que nous devrions garder un suivi plus récurrent sur nos avancement du projet. Un meilleur suivi des tâches avec Trello nous aurait permis de se situer plus efficacement dans le temps par rapport aux tâches à accomplir. En effet, en sépaparant les grandes tâches du projet en sous-tâches dans le Trello nous aurait permis de prendre consciences de l'ampleur des tâches à accomplir par rapport au temps restant avant les différents livrables. Pour ce qui est du prototype, la précision du lanceur de balle serait à améliorer. De plus, le poids du robot serait aussi à optimiser puisqu'il était quand même lourd. 

\subsection{Amélioration possibles}


Avec plus de temps, nous rajouterions une communication Bluetooth entre les verres et le robot comme planifié en début de session.
De plus, nous développerions notre propre application pour contrôler le robot au lieu de dépendre de l’application \emph{MIT App Inventor} actuellement utilisé. Ensuite, un système de lancement à deux roues améliorerait peut-être la précision du lancer.
