\section{Manuel d’utilisateur}
\todoWho{Justin}

\todoGeneric{2 à 4 pages de manuel}

\todoGeneric[inline]{Intro de la section manuel}

\begin{wrapfigure}{r}{0.25\linewidth}
    \centering
    \includegraphics[width=\linewidth]{example-image-a}
    \caption{Version finale du prototype}
    \label{fig:a1-protoype}
\end{wrapfigure}

Le robot est un lanceur de balle de bière-pong qui a comme utilisation première de lancer des balles dans des verres.
Le robot sera contrôlé par un utilisateur à l’aide d’une application mobile qui lui permettra de le faire bouger de gauche à droite et de contrôler la force du moteur avec de lancer la balle.
De plus, des verres intelligents accompagne le robot.
Les verres capte les détectes les balles à l’intérieur du verre pour avertir au joueur quels verres sont réussis à l’aide de lumières qui s’allument.

\subsection{Consignes d'utilisation}

\paragraph{Installation de l’application}

\begin{wrapfigure}{r}{0.5\linewidth}
    \centering

    \begin{subfigure}{.8\linewidth}
        \centering
        \includegraphics[width=\linewidth]{example-image-a}
        \caption{Construction du code QR}
        \label{fig:a1-create-code-QR}
    \end{subfigure}
    \begin{subfigure}{0.4\linewidth}
        \centering
        \includegraphics[width=\linewidth]{example-image-b}
        \caption{Code QR}
        \label{fig:a1-code-QR}
    \end{subfigure}
    \begin{subfigure}{0.4\linewidth}
        \centering
        \includegraphics[width=\linewidth]{example-image-c}
        \caption{Icône de l’application}
        \label{fig:a1-app-icon}
    \end{subfigure}

    \caption{Installation de l’application}
    \label{fig:template-example-flottante}
\end{wrapfigure}
\todo{add pictures}

\begin{enumerate}
    \item Créer un compte \href{https://appinventor.mit.edu/}{MIT App Inventor (https://appinventor.mit.edu/)}.
    \item Obtenir l’application sur son compte MIT App Inventor.
    \item Construire le code QR de l’application à l’aide d’un outil de MIT App Inventor (figure \ref{fig:a1-create-code-QR})
    \item Avec un téléphone Android, numériser le code QR, afin que l’application s’installe sur votre téléphone. (figure \ref{fig:a1-code-QR})
    \item Vous pouvez maintenant lancer l’application à l’aide de l’icône qui s’est installer sur votre téléphone. (figure \ref{fig:a1-app-icon})
\end{enumerate}

\paragraph{Démarrage du robot}

\begin{wrapfigure}{r}{0.5\linewidth}
    \centering

    \begin{subfigure}{0.4\linewidth}
        \centering
        \includegraphics[width=\linewidth]{example-image-a}
        \caption{Interrupteur de mise en marche}
        \label{fig:a1-interrupteur}
    \end{subfigure}
    \begin{subfigure}{0.4\linewidth}
        \centering
        \includegraphics[width=\linewidth]{example-image-b}
        \caption{Chargeur de batterie}
        \label{fig:a1-chargeur-batterie}
    \end{subfigure}

    \caption{Installation de l’application}
    \label{fig:template-example-flottante}
\end{wrapfigure}
\todo{add pictures}

\begin{enumerate}
    \item Soulever le chapeau du robot 
    \item Pour mettre en marche le robot il faut activer l’interrupteur. (figure \ref{fig:a1-interrupteur})
    \item Si le robot manque de puissance il faut penser à recharger la batterie qui se situe en dessous du robot. (figure \ref{fig:a1-chargeur-batterie})
    \item Remplir le réservoir du robot, afin de pouvoir lancer des balles.
    \item Placer le robot en face des verres à une distance raisonnable.
\end{enumerate}

\paragraph{Démarrage de la plaque des verres}

\begin{wrapfigure}{r}{0.3\linewidth}
    \centering
    \includegraphics[width=\linewidth]{example-image-a}
    \caption{Interrupteur de mise en marche de la plaque de verres}
    \label{fig:a1-interrupteur-verres}
\end{wrapfigure}
\todo{add pictures}

\begin{enumerate}
    \item Il suffit d’activer les deux interrupteurs sur la boite noire de la plaque des verres. (figure \ref{fig:a1-interrupteur-verres})
    \item Si les lumières n’allument pas, il faut changer les batteries (2 batteries 9V) dans la boite noire coller au module des verres.
\end{enumerate}

\paragraph{Connection de l'application au robot}

\todo{add pictures}

\begin{enumerate}
    \item Aller dans vos réglage Bluetooth de votre téléphone pour vous connecter au périphérique : HC05-Slave (Figure 8)
    \item Rentrer le mot de passe : 1234
    \item Aller dans l’application pour connecter le Bluetooth de l’application avec votre Bluetooth du téléphone (Figure 9)
    \item Quand le robot a été arrêter vous devez découpler HC05-Slave de vos périphériques déjà couplés et répéter les étape 1 à 3. (Figure 10 7)
\end{enumerate}

\paragraph{Contrôle du robot avec l’application}

\todo{add 2 pictures}

\begin{enumerate}
    \item Pour faire bouger le robot à gauche utiliser le bouton en haut à gauche de l’application (Figure 10 1)
    \item Pour faire bouger le robot à droite utiliser le bouton en haut à droite de l’application (Figure 10 2)
    \item Pour faire augmenter la puissance du moteur utiliser le bouton d’addition (Figure 10 3)
    \item Pour diminuer la puissance du moteur utiliser le bouton de soustraction (Figure 10 4)
    \item Pour lancer la balle utiliser le bouton en forme de balle de ping-pong (Figure 10 6)
\end{enumerate}
