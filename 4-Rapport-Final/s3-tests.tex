\section{Tests}

Tout au long de la conception du prototype, plusieurs tests ont été effectués.
Parmi ceux-ci, nous avons effectué des tests unitaires, des tests d'intégration et des tests des cas limites.
Les tests les plus pertinents sont présentés dans les paragraphes qui suivent.

\subsection{Tests unitaires}

Au cours de ce projet, nous avons fait plusieurs tests afin d’assurer le bon fonctionnement de notre prototype.
Bien sûr, avant d’avoir un prototype fini, il faut créer plusieurs modules de fonctionnalité séparément avant de les intégrer les uns aux autres.
De ce fait, nous avons alors fait des tests unitaires, c’est-à-dire des tests concernant individuellement chacun des modules de fonctionnalité.
Le premier module est le lanceur de balle de ping-pong.
Le deuxième est le système de verres intelligents.
Le suivant est le système Bluetooth reliant le robot à une application et finalement il y a le robot lui-même.
Deux moteurs sont installés sur le lanceur et ceux-ci sont testés séparément.
Un des principaux tests du lanceur était la tension d’alimentation du moteur sans balais.
Tel que démontré dans le tableau \ref{tab:s3-test-unitaires}, le courant minimum pour faire tourner le moteur est de 1,7~A.
D'autres tests unitaires du lanceur ont été effectués tel que la rotation du servomoteur permettant d’actionner le système de chargement des balles.
Des exemples de tests unitaires sont présentés dans le tableau \ref{tab:s3-test-unitaires}.

Notre moteur sans balais doit être alimenté à 7,4~V, mais sur le robot les seules sorties de tension sont de 5~V ou 12~V.
Nous avons alors ajusté notre régulateur de tension en tournant le potentiomètre pour qu’il convertisse la tension de 12~V en tension de 7,4~V.
Comme il est possible de voir dans le tableau ci-dessus, la plupart des tests étaient concluant puisque notre plan de tests unitaires étaient simples et efficace.
Les tests effectués étaient précis et faciles à réaliser ce qui nous a permis d’avoir des aussi bons résultats.
D'autres tests unitaires ont été faits par exemple le déplacement du robot.
Notre robot peut se déplacer d’avant ou d’arrière, mais il fallait savoir si ce déplacement était plus ou moins droit.
Lors du test, nous avons remarqué que le poids du lanceur sur la petite roue arrière créait le blocage de celle-ci lorsqu’elle tourne sur elle-même et donc lorsque le robot arrête et repart, cette roue crée un déplacement latéral indésirable qui change la trajectoire du robot.
Pour remédier à la situation, nous avons installé un support qui empêche la roue de tourner sur elle-même ce qui permet au robot de tenir une trajectoire plus droite.
Puisque le robot et le système de verre sont éloignés d’environ 2,5~mètres, un test sur la vitesse du moteur permettant à la balle d’atteindre les verres a été fait.
Nous savons que le temps minimal des impulsions envoyées à l'\emph{ESC} pour que le moteur puisse lancer une balle est de 1100~$\mu$s et qu’à 1150~$\mu$s le robot lance au-dessus de 3~mètre.
Nous voulions avoir environ 10 vitesses possibles pour l’utilisateur, alors nous avons décidé de faire des bonds de 5~$\mu$s entre ces deux valeurs.

\subsection{Tests d'intégration}

Ensuite, des tests intégrant tous les modules ensemble ont été effectués afin de s’assurer que tout fonctionne lorsque le prototype est terminé.
Des exemples de tests d'intégration sont présentés dans le tableau \ref{tab:s3-test-integration}.

\subsection{Tests des cas limites}

Les derniers tests effectués sont des tests de cas limite, c’est-à-dire les limites du prototype.
Nous avons testé la largeur d’impulsion minimale et maximale à envoyer à l'\emph{ESC} du moteur pour lancer une balle, le nombre de balles pouvant être lancé sans remplir le réservoir, la distance maximale entre l’application et le robot afin que le robot effectue encore les commandes demandées, la tension minimale pour allumer une DEL et la hauteur maximale et minimal de la roue propulsant la balle.
Des exemples de tests de cas limites sont présentées dans le tableau \ref{tab:s3-test-cas-limites}.

\begin{landscape}
    \begin{table}[h!]
        \centering
        \caption{Exemple de plan de tests unitaires}
        \begin{tabular}{p{0.5in}p{2in}p{1.5in}p{0.75in}p{3in}}
            \hline
            \bfseries Test & \bfseries Description & \bfseries Résultats attendus & \bfseries Réussite & \bfseries Justification \\
            \hline\hline
            1 & Courant du moteur sans balais & Au-dessous de 2,5~A & Oui & Le moteur accepte un courant de 1,7~A \\
            2 & Réception du signal de l’application par le récepteur Bluetooth & Le port série reçoit les chiffres affiliés à chaque bouton de l’application & Oui & Chaque bouton envoie un chiffre entre 1 et 6 qui est reçu par le récepteur \\
            3 & Tension de la DEL & Au-dessus de 0,5~V & Non & Nous avons obtenu des valeurs entre 0,4~V et 0,6~V \\
            \hline
        \end{tabular}
        \label{tab:s3-test-unitaires}
    \end{table}

    \begin{table}[h!]
        \centering
        \caption{Exemple de plan de tests d'intégration}
        \begin{tabular}{p{0.5in}p{2in}p{1.5in}p{0.75in}p{3in}}
            \hline
            \bfseries Test & \bfseries Description & \bfseries Résultats attendus & \bfseries Réussite & \bfseries Justification \\
            \hline\hline
            1 & Les fonctions de l’application s’effectuent sur le robot. & Lorsqu’on appuie sur une touche de l’application, le robot effectue la fonction. & Oui & Lorsqu’un appuie sur une touche de l’application, le robot effectue cette tâche c’est-à-dire qu’il tire quand on lui demande, il bouge quand on lui demande, etc. \\
            2 & Illumination des DEL & Lorsqu’il y a une balle dans un verre, la DEL reliée à ce verre s’allume. & Oui & La balle vient couper le faisceau infrarouge donc l’ampli-op ne reçoit plus de tension à sa borne négative alors la borne positive devient plus grande et donc l’amplificateur envoie une tension de 5~V. \\
            3 & Constance des tirs & Les tirs à une même vitesse devraient tous se retrouver dans une région circulaire d’environ 6~pouces de rayon & Non & Lors des tests effectués, le lanceur avait beaucoup de difficultés à réaliser les attentes. Il était plus ou moins constant, mais pas dans la région voulue. Pour remédier à ce problème, nous avons installé un canon plus long au lanceur. \\
            \hline
        \end{tabular}
        \label{tab:s3-test-integration}
    \end{table}

    \begin{table}[h!]
        \centering
        \caption{Exemple de plan de cas limites}
        \begin{tabular}{p{0.5in}p{2in}p{1.5in}p{0.75in}p{3in}}
            \hline
            \bfseries Test & \bfseries Description & \bfseries Résultats attendus & \bfseries Réussite & \bfseries Justification \\
            \hline\hline
            1 & Impulsion maximale pour propulser une balle & 2000~$\mu$s & Non & Après 1500~$\mu$s, la balle n’est pas vraiment lancée, elle agit aléatoirement. \\
            2 & Nombre de balles pouvant être lancé sans avoir à remplir le réservoir à balle & Environ 10 balles & Oui & Le réservoir peut contenir 10 balles \\
            3 & Tension minimale de la batterie des verres & 6,5~V & Oui & Si la batterie de 9~V se rend en dessous de 6,5~V, le système de verre ne fonctionne plus, car la tension de référence des régulateurs est de 6,5~V \\
            4 & Hauteur maximale de la roue propulsant la balle & 40~mm de la rampe de lancement & Oui & Au-dessus de 40 mm la roue ne touche pas à la balle, donc la balle n’est pas propulsée. \\
            \hline
        \end{tabular}
        \label{tab:s3-test-cas-limites}
    \end{table}
\end{landscape}
