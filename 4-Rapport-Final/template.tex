\section{Exemple de titre de section}

\todoGeneric[inline]{Exemple de description du contenu du paragraphe (NE PAS SUPPRIMER)}
Write actual content here.
You can split long line.
Like that.

You can also create new paragraph by adding a blank line.

\subsection{Exemples de TODO}

\todoPoints{Exemple de description du contenu du paragraphe (NE PAS SUPPRIMER)}
\todoDoing{Changer la ligne \texttt{\textbackslash todoPoints} en \texttt{\textbackslash todoDoing} avant de travailler sur un paragraph.}
\todoDone{Écrivez en \texttt{\textbackslash todoDone} lorsque que vous croyez que le paragraphe est terminé.}
\missingfigure{Inscrivez \texttt{\textbackslash missingfigure} si vous voules insérer une figure}

\section{Ajouter un tableau}

\begin{table}[h!]
    \centering
    \begin{tabular}{lllll}
        \hline
        Col1    & Col2    & Col3    & Col4    & Col5\\
        (unité) & (unité) & (unité) & (unité) & (unité)\\
        \hline\hline
        data & data & data & data & data\\
        data & data & data & data & data\\
        data & data & data & data & data\\
        data & data & data & data & data\\
        \hline
    \end{tabular}
    \caption{Titre du \textsubscript{super} tableau}
    \label{tab:template-Vout}
\end{table}

\section{Ajouter une citation}

On peut citer
un texte \cite{boutin_en_2009},
une figure (\ref{fig:template-example-image}),
une figure flottante (figure \ref{fig:template-example-flottante}) ou
un tableau \ref{tab:template-Vout}.

\section{Ajouter une image}

\begin{figure}[h!]
    \centering
    \includegraphics[width=0.8\linewidth]{example-image-a}
    \caption{Description de l'image.}
    \label{fig:template-example-image}
\end{figure}

\section{Ajouter une image flottante}

\begin{wrapfigure}{r}{0.25\linewidth}
    \centering
    \includegraphics[width=\linewidth]{example-image-a}
    \caption{Description de l'image.}
    \label{fig:template-example-flottante}
\end{wrapfigure}

Texte après
