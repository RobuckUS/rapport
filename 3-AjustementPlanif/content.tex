\section{Comparaison}

\todo{Add content}

\subsection{Division de tâches}

Il y a eu une amélioration dans la division des tâches. Dans le défi du parcours, 2 groupes de 2 personnes travaillaientt sur les mêmes tâches alors que les 3 autres personnes travaillaient sur le projet final. Cette façon de diviser les tâches n'est pas efficace car plusieurs individus travaillent en compétition plûtot que de s'entraider. Lors de l'épreuve du combattant, chaque membre de l'équipe a travaillé sur des tâches différentes pour progresser plus rapidement dans l'épreuve. Par exemple, un membre plus expérimenté en dessin 3D s'est chargé de faire la conception mécanique nécessaire pour le défi. Aussi, les responsables informatiques se sont occupé de la programmation.  Donc, notre méthode de travail de l'épreuve du combattant s'est avéré plus efficace que celle du défi du parcours.

\subsection{Travaux prévus versus effectués}

Avant le défi du parcours, nous avons planifié plusieurs travaux pour avancer le projet final. La plupart d'entre-eux ont été effectué avec succès. Par contre, certain travaux comme la télécommande n'ont pas été completés au moment prévu et devront être reporté.

\subsection{Problèmes rencontrés et solutions trouvées}

\todo{Add content}

\subsection{Points positifs et négatifs de votre planification}

\todo{Add content}

\subsection{Améliorations futures à votre gestion de projet}

Une amélioration a apporté serait de maintenir un rythme de travail plus constant. De plus, les réunions hebdomadaires de suivis devrait être maintenues au lieu d'être annuler.
