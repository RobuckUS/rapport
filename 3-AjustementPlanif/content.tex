\section{Division des tâches}

\begin{spacing}{1.5}
    Il y a eu une amélioration dans la division des tâches.
    En effet, dans le défi du parcours, quatre personnes, divisées en deux groupes de deux personnes, travaillaient sur les mêmes tâches alors que les trois autres personnes travaillaient sur le projet final.
    Cette façon de diviser les tâches n'est pas efficace, car plusieurs individus travaillent en compétition plutôt qu'en collaboration.
\end{spacing}

\begin{spacing}{1.5}
    Lors de l'épreuve du combattant, chaque membre de l'équipe a travaillé sur des tâches différentes afin de progresser plus rapidement dans l'épreuve.
    Par exemple, un membre plus expérimenté en dessin 3D s'est chargé de faire la conception mécanique nécessaire pour le défi et les responsables informatiques se sont occupés de la programmation.
\end{spacing}

\begin{spacing}{1.5}
    Nous pouvons donc constater que la division des tâches durant la préparation de l'épreuve du combattant s'est avérée plus efficace que pendant la préparation du défi du parcours.
\end{spacing}

\section{Travaux prévus par rapport aux travaux effectués}

\begin{spacing}{1.5}
    Avant le défi du parcours, nous avons planifié plusieurs travaux pour avancer le projet final.
    La plupart d'entre-eux ont été effectués avec succès.
    Par contre, certains travaux, tel que la réalisation de la télécommande, n'ont pas été terminés au moment prévu et devront être reportés.
\end{spacing}

\begin{spacing}{1.5}
    Tous les travaux prévus pour l'épreuve du combattant ont été effectués avec succès, mais pas toujours dans le temps prévus.
\end{spacing}

\begin{spacing}{1.5}
    Bref, nous constatons que le ratio entre les travaux effectués et planifiés est similaire entre les deux défis et, par le fait même, qu'il est demeuré stable entre les deux défis.
\end{spacing}

\pagebreak

\section{Aspects positifs et négatifs de notre planification}

\begin{spacing}{1.5}
    Dans notre planification, nous avons prévu des périodes sans tâches pour approfondir les tests ou pour absorber les délais accumulés.
    Ces périodes ont fait en sorte que nous avons réussi à finir les deux défis malgré la procrastination des membres de l'équipe.
\end{spacing}

\begin{spacing}{1.5}
    Dans un autre ordre d'idée, notre planification avait aussi quelques lacunes.
    En effet, nous avons planifié certains travaux très long sans les séparer en plus petit morceaux.
    De plus, nous avons négligé d'inclure certaines tâches dont on croyait que la durée était négligeable.
\end{spacing}

\begin{spacing}{1.5}
    N'ayant pas pris le temps d'effectuer des changements entre les deux défis, nous avons reproduit les mêmes lacunes dans la planification de l'épreuve du combattant.
\end{spacing}

\section{Améliorations futures à votre gestion de projet}

\begin{spacing}{1.5}
    Pour le projet final, nous aimerions corriger les lacunes que nous avons identifiées dans notre planification et maintenir les correctifs que nous avons apportés au cours des deux premiers défis.
\end{spacing}

\begin{spacing}{1.5}
    D'abord, les tâches planifiées devraient être moins grosses et plus précises.
    Cela nous permettrait de prédire avec plus de précision la charge de travail réel nécessaires pour réaliser les différents travaux.
\end{spacing}

\begin{spacing}{1.5}
    Ensuite, les réunions hebdomadaires de suivis devraient être maintenues, même si peu de tâches ont été effectués au cours de la semaine précédente.
    Cela nous permettrait de comprendre les raisons pour lesquelles nous avons moins travaillé et mettre en place les mesures nécessaires pour éviter que cela ne se reproduise.
\end{spacing}

\begin{spacing}{1.5}
    Enfin, nous souhaiterions maintenir un rythme de travail plus constant.
\end{spacing}
