\section{Comparaison}

\subsection{Division de tâches}

\begin{spacing}{1.5}
    Il y a eu une amélioration dans la division des tâches. Dans le défi du parcours, 2 groupes de 2 personnes travaillaient sur les mêmes tâches alors que les 3 autres personnes travaillaient sur le projet final. Cette façon de diviser les tâches n'est pas efficace, car plusieurs individus travaillent en compétition plutôt que de s'entraider. Lors de l'épreuve du combattant, chaque membre de l'équipe a travaillé sur des tâches différentes pour progresser plus rapidement dans l'épreuve. Par exemple, un membre plus expérimenté en dessin 3D s'est chargé de faire la conception mécanique nécessaire pour le défi. Aussi, les responsables informatiques se sont occupés de la programmation.  Donc, notre méthode de travail de l'épreuve du combattant s'est avérée plus efficace que celle du défi du parcours.
\end{spacing}

\subsection{Travaux prévus versus effectués}

\begin{spacing}{1.5}
    Avant le défi du parcours, nous avons planifié plusieurs travaux pour avancer le projet final. La plupart d'entre-eux ont été effectué avec succès. Par contre, certains travaux comme la télécommande n'ont pas été completés au moment prévu et devront être reportés. Tous les travaux prévus pour l'épreuve du combattant et le défi du parcours ont été effectués avec succès, mais pas toujours dans le temps prévus. Le ratio entre les travaux effectués et planifiés sont similaires entre les 2 défis.
\end{spacing}

\subsection{Points positifs et négatifs de votre planification}

\begin{spacing}{1.5}
    Dans notre planification, nous avons prévu des périodes sans tâches pour approfondir les tests ou pour absorber les délais accumulés. Le positif dans cette planification est que nous avons réussi à finir les deux défis malgré notre procrastination. Par contre, notre planification avait de trop grosses tâches. De plus, certaines tâches dont on croyait que la durée allait être négligeable n'apparaissaient pas dans celle-ci.
\end{spacing}

\subsection{Améliorations futures à votre gestion de projet}

\begin{spacing}{1.5}
    Une amélioration a apporté serait de maintenir un rythme de travail plus constant. De plus, les réunions hebdomadaires de suivis devraient être maintenues au lieu d'être annuler. Les tâches planifiées à l'aide de Trello devraientt être moins grosses et plus précises.
\end{spacing}
