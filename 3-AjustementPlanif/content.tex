\section{Comparaison}

\subsection{Division des tâches}

\begin{spacing}{1.5}
    Il y a eu une amélioration dans la division des tâches.
    En effet, dans le défi du parcours, quatres personnes, divisées en deux groupes de deux personnes, travaillaient sur les mêmes tâches alors que les trois autres personnes travaillaient sur le projet final.
    Cette façon de diviser les tâches n'est pas efficace, car plusieurs individus travaillent en compétition plutôt qu'en collaboration.
    Lors de l'épreuve du combattant, chaque membre de l'équipe a travaillé sur des tâches différentes afin de progresser plus rapidement dans l'épreuve.
    Par exemple, un membre plus expérimenté en dessin 3D s'est chargé de faire la conception mécanique nécessaire pour le défi et les responsables informatiques se sont occupés de la programmation.
    Nous pouvons donc constater que la division des tâches durant la préparation de l'épreuve du combattant s'est avérée plus efficace que pendant la préparation du défi du parcours.
\end{spacing}

\subsection{Travaux prévus par rapport aux travaux effectués}

\begin{spacing}{1.5}
    Avant le défi du parcours, nous avons planifié plusieurs travaux pour avancer le projet final.
    La plupart d'entre-eux ont été effectué avec succès.
    Par contre, certains travaux, tel que la réalisation de la télécommande n'ont pas été completés au moment prévu et devront être reportés.
    Tous les travaux prévus pour l'épreuve du combattant ont été effectués avec succès, mais pas toujours dans le temps prévus.
    Bref, nous constatons que le ratio entre les travaux effectués et planifiés est similaire entre les deux défis et, par le fait même, qu'il est demeuré stable entre les deux défi.
\end{spacing}

\subsection{Aspects positifs et négatifs de notre planification}

\begin{spacing}{1.5}
    Dans notre planification, nous avons prévu des périodes sans tâches pour approfondir les tests ou pour absorber les délais accumulés.
    Ces périodes ont fait en sorte que nous avons réussi à finir les deux défis malgré la procrastination des membres de l'équipe.
    Dans un autre ordre d'idée, notre planification avait aussi quelques lacunes. En effet, nous avons planifié certaines travaux très long sans les séparer en plus petit morçeaux.
    De plus, nous avons négligé d'inclure certaines tâches dont on croyait que la durée était négligeable.
    N'ayant pas pris le temps d'effectuer des changements entre les deux défis, nous avons reproduit les mêmes lacunes dans la planification de l'épreuve du combattant.
\end{spacing}

\subsection{Améliorations futures à votre gestion de projet}

\begin{spacing}{1.5}
    Une amélioration a apporté serait de maintenir un rythme de travail plus constant.
    De plus, les réunions hebdomadaires de suivis devraient être maintenues au lieu d'être annuler.
    Les tâches planifiées à l'aide de Trello devraientt être moins grosses et plus précises.
\end{spacing}
